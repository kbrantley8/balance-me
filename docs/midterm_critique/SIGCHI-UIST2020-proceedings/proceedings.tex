\documentclass{sigchi}

% Copyright
\CopyrightYear{2020}
\setcopyright{acmlicensed}

% Use this command to override the default ACM copyright statement
% (e.g. for preprints).  Consult the conference website for the
% camera-ready copyright statement.

\toappear{
Permission to make digital or hard copies of all or part of this work
for personal or classroom use is granted without fee provided that
copies are not made or distributed for profit or commercial advantage
and that copies bear this notice and the full citation on the first
page. Copyrights for components of this work owned by others
must be honored. Abstracting with credit is permitted.

\medskip
CS 4605, Summer 2020

Georgia Institute of Technology

Atlanta, GA 30332
}

% Load basic packages
\usepackage{balance}       % to better equalize the last page
\usepackage{graphics}      % for EPS, load graphicx instead 
\usepackage[T1]{fontenc}   % for umlauts and other diaeresis
\usepackage{txfonts}
\usepackage{mathptmx}
\usepackage[pdflang={en-US},pdftex]{hyperref}
\usepackage{color}
\usepackage{booktabs}
\usepackage{textcomp}

% Some optional stuff you might like/need.
\usepackage{microtype}        % Improved Tracking and Kerning
% \usepackage[all]{hypcap}    % Fixes bug in hyperref caption linking
\usepackage{ccicons}          % Cite your images correctly!
% \usepackage[utf8]{inputenc} % for a UTF8 editor only

% If you want to use todo notes, marginpars etc. during creation of
% your draft document, you have to enable the "chi_draft" option for
% the document class. To do this, change the very first line to:
% "\documentclass[chi_draft]{sigchi}". You can then place todo notes
% by using the "\todo{...}"  command. Make sure to disable the draft
% option again before submitting your final document.
\usepackage{todonotes}

% Paper metadata (use plain text, for PDF inclusion and later
% re-using, if desired).  Use \emtpyauthor when submitting for review
% so you remain anonymous.
\def\plaintitle{Balance Me: Midterm Critique}
\def\plainauthor{Kory Brantley, Dylan Cripe, Morgan Gomez, Rahul Patel,
and Alex Schwartz}
\def\emptyauthor{}
\def\plainkeywords{Authors' choice; of terms; separated; by
  semicolons; include commas, within terms only; this section is required.}
\def\plaingeneralterms{Documentation, Standardization}

% llt: Define a global style for URLs, rather that the default one
\makeatletter
\def\url@leostyle{%
  \@ifundefined{selectfont}{
    \def\UrlFont{\sf}
  }{
    \def\UrlFont{\small\bf\ttfamily}
  }}
\makeatother
\urlstyle{leo}

% To make various LaTeX processors do the right thing with page size.
\def\pprw{8.5in}
\def\pprh{11in}
\special{papersize=\pprw,\pprh}
\setlength{\paperwidth}{\pprw}
\setlength{\paperheight}{\pprh}
\setlength{\pdfpagewidth}{\pprw}
\setlength{\pdfpageheight}{\pprh}

% Make sure hyperref comes last of your loaded packages, to give it a
% fighting chance of not being over-written, since its job is to
% redefine many LaTeX commands.
\definecolor{linkColor}{RGB}{6,125,233}
\hypersetup{%
  pdftitle={\plaintitle},
% Use \plainauthor for final version.
%  pdfauthor={\plainauthor},
  pdfauthor={\emptyauthor},
  pdfkeywords={\plainkeywords},
  pdfdisplaydoctitle=true, % For Accessibility
  bookmarksnumbered,
  pdfstartview={FitH},
  colorlinks,
  citecolor=black,
  filecolor=black,
  linkcolor=black,
  urlcolor=linkColor,
  breaklinks=true,
  hypertexnames=false
}

% create a shortcut to typeset table headings
% \newcommand\tabhead[1]{\small\textbf{#1}}

% End of preamble. Here it comes the document.
\begin{document}

\title{\plaintitle}

\numberofauthors{5}
\author{
\plainauthor
%  \alignauthor{Kory Brantley\\
%    \affaddr{CS (HCI/UI)}\\
%    \email{kbrantley@gatech.edu}}\\
%  \alignauthor{Dylan Cripe\\
%    \affaddr{CS (Intelligence/HCI)}\\
%    \email{dcripe@gatech.edu}}\\
%  \alignauthor{Morgan Gomez\\
%    \affaddr{CM}\\
%    \email{mgomez46@gatech.edu}}\\
%  \alignauthor{Rahul Patel\\
%    \affaddr{CS (Info Internetworks/Devices)}\\
%    \email{rahul199722@gmail.com}}\\
%  \alignauthor{Alex Schwartz\\
%    \affaddr{CompE}\\
%    \email{aschwartz@gatech.edu}}\\
}

\maketitle

\section{Introduction}

The digital world in which we now live has given us access to an exceedingly
large amount of information at our fingertips. While beneficial, this vast
amount of information and the constant connection to it may be distracting to
teenagers, especially those with intellectual or developmental disabilities
(IDD). There are approximately 6.5 million people with an intellectual or
developmental disability in the United States [1], and they may have difficulty
balancing their time between required tasks and recreational ones. While there
are multiple solutions currently on the market for keeping on task, most of
them are either too simple (i.e., countdown clocks) or too complex. The goal of
this project is to develop an interactive and minimalist smartphone application
and companion smartwatch application to help teenagers with and without IDD
manage their required work while still making time to do things they want to do.

\section{Related Work}

Currently, there are many different types of time management and task tracking
applications on the market; some specialize in scheduling, habits, and routines
while others emphasize focus and what the user is spending their time on.
\textit{Focus Keeper} is a mobile app that utilizes an onscreen timer to keep
users focused and allows users to schedule breaks and set session lengths [2].
\textit{Toggl} is another mobile application that allows users to track and
monitor the time they spend on activities, while also setting goals and plans
for completion [3]. Applications like \textit{Stepping Stones} and
\textit{CanPlan} are geared towards users with disabilities, both creating
routines for users to modify and stick to [4, 5]. Additionally,
\textit{Stepping Stones} consists of very detailed steps in which a user should
complete each day [4].

What separates \textit{Balance Me} from other existing technologies is the
combination of individual and team task management, along with a wearable
application to ensure that time management tasks are completed. Those using our
application are able to view all tasks in one place in a format that is easy to
use. Existing applications geared towards time management are often either too
complicated for adolescents with intellectual disabilities, or too simple to be
stimulating and useful long term. For instance, the application
\textit{Focus Keeper} [2] encourages users to focus via starting and stopping a
timer, but fails to incorporate scheduling of future tasks or the creation of
different types of tasks. Time management technologies geared towards those with
developmental disabilities, such as \textit{Task Timer} [6], are too simple to
be used by those with and without intellectual disabilities. In comparison,
\textit{Monday.com} [7] provides a solution to group task management and allows
for the managing, creation, and scheduling of tasks, but is too complicated for
individuals looking for a simplified time management tool. \textit{Balance Me}
aims to provide a middle ground in the form of an easily navigable application
that incorporates existing wearable technology to reinforce time management
goals.

\section{Current Work}

Our team has divided the work on the mobile application into front-end
development and back-end development. This section talks about the work that
has been done thus far on both fronts.

\subsection{Front-End Development}

With target users in mind, the focus of \textit{Balance Me} is to deliver a
user interface that is easy to use and motivating The user should feel
accomplished when they finish a task, but also, should not feel unmotivated
when they fail to finish a task on time. The purpose of this application is to
help the user enhance their time and task management skills. If a task is not
completed on time, the user should learn from the failure rather than feel
defeated.

Currently, the \textit{Balance Me} team has established the groundwork to
develop these functions. Our focus for the first half of the semester was to
understand and develop the flow of the application, which supports the user’s
schedule rather than dictate it. For example, the user has an option to create a
task, but not schedule it. This provides the user with certain flexibility
while using the application. The current UI is designed to be minimal yet
impactful. We do not want to overwhelm users with information.


\subsection{Back-End Development}

Our goal for the backend design was to create a RESTful API that had direct
communication with our database. To do this, we used Express, Node.js, and
Mongoose for our server communication. For our database, we are using MongoDB
Atlas, as it provides us a database with lenient reading/writing and plenty of
space to hold our data. Furthermore, we are using Heroku to host our backend
repository, which lets us make calls to our API from any device.

The reason we chose to use Mongoose is its models. Mongoose models utilize
schemas to let us easily layout what data we want each object type to
encapsulate. For the \textit{Balance Me} app, this is the creation of two main
object types: User and Task. The user model holds information about each
individual user’s data, such as their first name and their email:

In contrast to the simple user model, we created a task model that would be
capable of dynamic updates and easy repetition:

Each task is referring to the user who created it and the user it is assigned
to. This allows for the possibility of another ``parent'' account to create a
task and assign it to a ``child'' account. In the case of a caregiver and a
child who has a mental disability, this would allow for the caregiver to easily
create and assign a task to the child from anywhere.

\section{Discussion}

At the midway point in the semester, perhaps the most important thing we have
learned so  far is the ability to find the middle ground between our ideal goals
and reality. As a group, none of us had much experience with React Native, the
framework we are using to build our application. Therefore, it took about a
week for everyone to get comfortable programming and creating components,
screens, and other react native pieces required for this project. When writing
our proposal, our timeline seemed easy enough to stick to, but with the learning
curve required to do most of the work, we found ourselves about a week behind
because we did not factor in the time it takes to get used to the framework.
This meant that we had to revise our timeline and as a group decide which
aspects of our application are most important, and focus on those first before
getting into other additional features. 

Despite seeming like a negative, this actually allowed for our team to have
in-depth discussions about what elements of the application are most important
to the user and do more research on our user’s needs that we would not have done
otherwise. This also caused us to investigate what components we can outsource
to existing available components that we can then customize to fit our needs,
instead of building everything ourselves. Through these discussions, we have
also learned that our initial implementation of the project needed tweaking as
our core user group has different needs than the everyday individual, so despite
our initial excitement to add in a new feature or cool design to our
application, our focus needs to stay on our user and their experience. This
means that our  design needs to stay simple and focused on the application goals
instead of fun elements that while we would enjoy creating, do not adhere to the
project’s core goals. These elements are still included in our timeline, but are
now placed towards the end of the semester so that we can create them if we have
extra time towards the end of the course.


\section{Future Work}

Work on this project will continue to progress according to the following
updated timeline:

\newenvironment{myitemize}
{ \begin{itemize}
    \setlength{\itemsep}{0pt}
    \setlength{\parskip}{0pt}
    \setlength{\parsep}{0pt}     }
{ \end{itemize}                  } 

\begin{myitemize}
  \setlength{\itemsep}{0pt}
  \setlength{\parskip}{0pt}
  \setlength{\parsep}{0pt}
	\item \textbf{Week 4 (June 28--July 4): Mobile Improvements}
	\begin{myitemize}
		\item Continued development of backend/data storage
		\item Finalize flow of screens in mobile app
		\item As a user, I want to see graphs/charts to be able to monitor my progress
		\begin{myitemize}
			\item \textit{Feature:} Page to see statistics of previous tasks
		\end{myitemize}
		\item As a user, I want tips and how-to tutorials on how to do tasks
		\begin{myitemize}
			\item \textit{Feature:} Tips and how-to’s added to active task screen
		\end{myitemize}
	\end{myitemize}
  
	\item \textbf{Week 5 (July 5--11): Wearable Application Development and Gamification}
	\begin{myitemize}
		\item Adapt backend to wearable application
		\item As a user, I want to start a task from a wearable
		\begin{myitemize}
			\item \textit{Feature:} Wearable app creation of a task
		\end{myitemize}
		\item As a user, I want to have default tasks available to choose from a quick start option on a wearable device
		\begin{myitemize}
			\item \textit{Feature:} Predefined quick start list
		\end{myitemize}
		\item As a user, I want a visual game component to motivate me to complete tasks on time
		\begin{myitemize}
			\item \textit{Feature:} Creation of game component to be integrated into the application
			\item \textit{Feature:} Page dedicated to see the game component and total points/level
		\end{myitemize}
	\end{myitemize}
	
	\item \textbf{Week 6 (July 12--18): Mobile and Wearable Improvements}
	\begin{myitemize}
		\item Continue implementation of visual game component
		\item As a user, I want to be notified for upcoming tasks from my wearable and phone
		\begin{myitemize}
			\item \textit{Feature:} Vibrations on wearable and notification on phone to tell the user when a task needs to be completed soon or if timer is almost done
		\end{myitemize}
		\item As a user, I want to share my progress with others in the form of game statistics
		\begin{myitemize}
			\item \textit{Feature:} Sharing stats via social media
			\item \textit{Feature:} Seeing other user’s progress as well as your own in the settings page
		\end{myitemize}
	\end{myitemize}
	
	\item \textbf{Week 7 (July 19--24): Extra Features \& Bug Fixes \textit{(Project due July 24)}}
	\begin{myitemize}
		\item Fix existing bugs in the applications
		\item As a user, I want to customize my game character
		\begin{myitemize}
			\item \textit{Feature:} Customization page
		\end{myitemize}
		\item As a user, I want to monitor my time spent on other applications
		\begin{myitemize}
			\item \textit{Feature:} Add setting button to monitor and notify user of excessive app usage
		\end{myitemize}
		\item As a user, I want the application to be visually appealing
		\begin{myitemize}
			\item \textit{Improvement:} UI Improvements (if needed)
		\end{myitemize}
		\item Project Deliverable
	\end{myitemize}

\end{myitemize}

\section{Conclusion}

With the development of \textit{Balance Me} we hope to create a product that
can help teenagers with and without IDD balance their time between required
tasks and leisure activities. We believe that the work that has been
accomplished so far is promising and will lead to an innovative solution for
this problem space. We believe that, by the end of the semester, we will have a
functioning product that enables users to improve their lives by successfully
managing their time.

\section{References}
[1] ``What is Intellectual Disability?'', \textit{SpecialOlympics.org}, 2020. [Online]. Available: https://www.specialolympics.org/about/intellectual-disabilities/what-is-intellectual-disability. [Accessed: 08- Jun- 2020].

[2] ``Focus Keeper - Time Management'', \textit{App Store}, 2020. [Online]. Available: https://apps.apple.com/us/app/focus-keeper-time-management/id867374917. [Accessed: 08- Jun- 2020].

[3] ``Toggl - Features'', \textit{Toggl.com}, 2020. [Online]. Available: https://www.toggl.com/
features/. [Accessed: 08- Jun- 2020].

[4] ``Stepping Stones - Daily Routines | Social App Hub'', \textit{Socialapphub.com}, 2020. [Online]. Available: https://www.socialapphub.com/app/stepping-stones-daily-routines. [Accessed: 08- Jun- 2020].

[5] ``CanPlan - a task manager app'', \textit{https://www.canassist.ca}, 2020. [Online]. Available: https://www.canassist.ca/EN/main/programs/technologies-and-devices/at-home/canplan.html. [Accessed: 08- Jun- 2020].

[6] ``A Task Timer for Special Education'', \textit{Resna.org}, 2020. [Online]. Available: https://www.resna.org/sites/default/files/legacy/conference/
proceedings/2003/Papers/TSP/Oleson\_TSP.htm. [Accessed: 08- Jun- 2020].

[7] ``Time Management | monday.com'', \textit{monday.com}, 2020. [Online]. Available: https://monday.com/lp/aw/timemgmt/remote. [Accessed: 08- Jun- 2020].

\end{document}